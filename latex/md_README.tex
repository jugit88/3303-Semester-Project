\subsection*{Team Members\+:}

James Draper, Drew Meyers, and Jeremy Udis. \subsection*{Github ID\textquotesingle{}s\+:}

jugit88, drewmeyerscuboulder, james3d

\subsection*{Title\+:}

Air Quality In\+Quiry (A\+Q\+IQ)\+: Community-\/led Air Quality Measurements and Analysis

\subsection*{Current U\+RL\+:}

\href{http://ec2-52-26-114-13.us-west-2.compute.amazonaws.com/login-site/login.php}{\tt project website}

\subsection*{N\+O\+TE}

We worked on this project as a group from a remote desktop. James just so happened to be signed into github in the shell, so all the commits went to him. WE A\+LL W\+O\+R\+K\+ED ON T\+H\+IS P\+R\+O\+J\+E\+CT T\+O\+G\+E\+T\+H\+ER. \subsection*{Description\+:}

The Hannigan research group at the University of Colorado Boulder has been developing low cost air quality monitoring devices called “pods.\+” These pods are small, light weight, and easily deployable to almost any location around the world; they offer numerous advantages over larger, more expensive equipment. For example, pods can be deployed indoors to study the effects of air quality on human health in enclosed environments. With pods, researchers and citizens alike can get key information on air quality and use it to study the impact of indoor air quality on human health. Additionally, data taken from pods can supplement other air quality source data to create better resolution.

Currently, the Hannigan lab does not provide a common place for their pod-\/users to store, share, and analyze pod data. Our team will create a user friendly web-\/interface for pod-\/users to store, share, and analyze air quality data collected by the pods.

\subsection*{Vision Statement\+:}

A\+Q\+IQ bridges the gap between citizens and scientists by providing communities with low-\/cost air quality data analysis tools.

\subsection*{Motivation\+:}

The primary motivation of this project is to gather, analyze, and present data in form that is visually pleasing and intuitive to use. In this process we will learn valuable skills such as using and gathering data from an arduino and pushing the data to a database use to analzye the data. As a group we wanted to know more about javascript and D3 in order to present the data in a unique way.

\subsection*{Risks}

We will be working with a data visualization library that none of us have used. Everyone in the group is pretty new to javascript. Meeting up as a group can be difficult.

\subsection*{Mitigation Strategy}

If there is a major problem we will meet as a group and reevaluate where we are and what we need to do.

\subsection*{Requirements\+:}

User Requirements\+: User will be able to upload a csv file which will then be stored into the database. We will then query the database to output a J\+S\+ON file in order to run Javascript based D3.

\subsection*{Functional Requirements\+:}

\tabulinesep=1mm
\begin{longtabu} spread 0pt [c]{*3{|X[-1]}|}
\hline
\rowcolor{\tableheadbgcolor}{\bf Id\+: }&\PBS\centering {\bf Description\+: }&\PBS\raggedleft {\bf Time(hours)  }\\\cline{1-3}
\endfirsthead
\hline
\endfoot
\hline
\rowcolor{\tableheadbgcolor}{\bf Id\+: }&\PBS\centering {\bf Description\+: }&\PBS\raggedleft {\bf Time(hours)  }\\\cline{1-3}
\endhead
1 &\PBS\centering As a learner, I want learn arduino, so that I can add wifi capabilities &\PBS\raggedleft 6 \\\cline{1-3}
2 &\PBS\centering As a developer, I want to query a database, so that I can gather data &\PBS\raggedleft 4 \\\cline{1-3}
3 &\PBS\centering As a developer, I want to learn the D3 library, so I can make visuals &\PBS\raggedleft 8 \\\cline{1-3}
4 &\PBS\centering As a developer, I want to learn Javacript, so I can use D3 &\PBS\raggedleft 8 \\\cline{1-3}
\end{longtabu}
\subsection*{Non-\/\+Functional Requirements\+:}

\tabulinesep=1mm
\begin{longtabu} spread 0pt [c]{*3{|X[-1]}|}
\hline
\rowcolor{\tableheadbgcolor}{\bf Id\+: }&\PBS\centering {\bf Description\+: }&\PBS\raggedleft {\bf Time(hours)  }\\\cline{1-3}
\endfirsthead
\hline
\endfoot
\hline
\rowcolor{\tableheadbgcolor}{\bf Id\+: }&\PBS\centering {\bf Description\+: }&\PBS\raggedleft {\bf Time(hours)  }\\\cline{1-3}
\endhead
11 &\PBS\centering As a team, we want to have a website to upload csv files. &\PBS\raggedleft 8 \\\cline{1-3}
12 &\PBS\centering As a team, we want query a database in order to make D3 visuals. &\PBS\raggedleft 6 \\\cline{1-3}
13 &\PBS\centering As a team, we want to work as a team to create a friendy environment. &\PBS\raggedleft NA \\\cline{1-3}
14 &\PBS\centering As a producer, I want to make a fast website, to improve user productivity &\PBS\raggedleft 6 \\\cline{1-3}
\end{longtabu}


\subsection*{Methodology}

We are going to use an Agile methodology to start.

\subsection*{Project Tracking Software}

\href{https://trello.com/aqiq}{\tt Trello}

\subsection*{Project Plan}

\href{https://trello.com/b/DfhxuGFe/project-plan}{\tt Layout} 